\section{模型假设与分析}
\begin{enumerate}
    \item 统计对象限制在成电清水河校区的研究生\\
    $\hookrightarrow $\textit{分析:本科生活动范围集中于校园南侧区域、较少有同学购买电动自行车;研究生教研室分布距离宿舍区普遍较远且自身财力更能支持其购买电动自行车,研究生群体中拥有电动车比例远高于本科生。} 
  
    \item 模型参考运输问题进行建立。\\
    $\hookrightarrow $\textit{分析:电动车充电需要从骑者地点运输至充电桩进行充电,可将双方看作产地与销地,费用以路程代替。至于‘产量’‘销量’的大小关系,可将其假设大小相同(免除虚拟地的假设).}\textbf{最终计算结果用以衡量比例大小关系而非绝对数值关系。} 

\end{enumerate}